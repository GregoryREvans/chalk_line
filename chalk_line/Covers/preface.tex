\documentclass[11pt]{article}
\usepackage{fontspec}
\usepackage[utf8]{inputenc}
\setmainfont{STIXGeneral}
\usepackage[paperwidth=21in,paperheight=13.59in,margin=1in,headheight=0.0in,footskip=0.5in,includehead,includefoot,portrait]{geometry}
\usepackage[absolute]{textpos}
\TPGrid[0.5in, 0.25in]{23}{24}
\parindent=0pt
\parskip=12pt
\usepackage{nopageno}
\usepackage{graphicx}
\graphicspath{ {./images/} }
\usepackage{amsmath}
\usepackage{tikz}
\newcommand*\circled[1]{\tikz[baseline=(char.base)]{
            \node[shape=circle,draw,inner sep=1pt] (char) {#1};}}

\begin{document}

\begin{textblock}{23}(0, 1)
\begin{center}
\huge FOREWORD
\end{center}
\end{textblock}

\vspace*{0.25\baselineskip}

\begingroup
\begin{center}
\leftskip5.5in
A Chalk Line is a device used for the creation of straight lines to aid in the construction of buildings. A length of string covered in chalk (or sometimes ink) powder is stretched across a flat surface, which is then pulled back and snapped onto the surface, leaving a straight line of the dust behind. This piece consists of three subsections entitles ``Calligraphy,'' ``Spirals,'' and ``Afterimage'' respectively.
\rightskip\leftskip
\phantom{text} \hfill (G.R.E.)
\end{center}
\endgroup

\vspace*{0.25\baselineskip}

\begingroup
\begin{center}
\leftskip3.5in
Lorem ipsum dolor sit amet, consectetur adipiscing elit. Morbi neque nibh, efficitur a eros at, elementum mollis nunc. Proin consectetur magna id purus euismod tristique. Aenean sit amet pulvinar nibh. Vivamus malesuada finibus pretium. Ut ligula mauris, commodo ac magna at, vestibulum accumsan purus. Vivamus tincidunt accumsan cursus. Ut arcu diam, vestibulum eget eros ultricies, sollicitudin facilisis tortor. Quisque massa arcu, dictum eget tristique sit amet, sagittis non nisl. Mauris varius lectus non nisl vehicula, ut feugiat ante accumsan. Nunc venenatis dui consequat erat porttitor, sit amet mollis nisl pharetra. Morbi hendrerit fringilla mauris, vitae rhoncus ante facilisis sit amet. Aenean erat felis, lobortis id dui sed, tempor finibus leo.
\rightskip\leftskip
\phantom{text} \hfill (Juanita Pineda)
\end{center}
\endgroup

\vspace*{1.25\baselineskip}

\begin{center}
\huge PERFORMANCE NOTES
\end{center}

\begingroup
\begin{center}

\leftskip2.25in
\pmb{Pitch} : Accidentals apply only to the pitch which they immediately precede, but persist through ties.
\rightskip\leftskip
\phantom{text} \hfill \phantom{()}

\vspace*{0.25\baselineskip}

\leftskip2.25in
\pmb{Miscellaneous} : \circled{1} Triangle note heads indicate \textit{pizzicato}. \circled{2} The symbol ``$\circ$" over a note represents a mostly airy tone-color that still retains some pitch. \circled{3} The symbol ``\o" \ represents a tone-color that is halfway between a normal playing technique and the $\circ$ technique.
\rightskip\leftskip
\phantom{text} \hfill \phantom{()}
\end{center}
\endgroup

\vspace*{15\baselineskip}

\begin{center}
c. 4'30''
\end{center}

\end{document}
