\documentclass[11pt]{article}
\usepackage{fontspec}
\usepackage[utf8]{inputenc}
\setmainfont{STIXGeneral}
\usepackage[paperwidth=21in,paperheight=13.59in,margin=1in,headheight=0.0in,footskip=0.5in,includehead,includefoot,portrait]{geometry}
\usepackage[absolute]{textpos}
\TPGrid[0.5in, 0.25in]{23}{24}
\parindent=0pt
\parskip=12pt
\usepackage{nopageno}
\usepackage{graphicx}
\graphicspath{ {./images/} }
\usepackage{amsmath}
\usepackage{tikz}
\newcommand*\circled[1]{\tikz[baseline=(char.base)]{
            \node[shape=circle,draw,inner sep=1pt] (char) {#1};}}

\begin{document}

\begin{textblock}{23}(0, 1)
\begin{center}
\huge FOREWORD
\end{center}
\end{textblock}

\vspace*{0.25\baselineskip}

\begingroup
\begin{center}
\leftskip5.5in
A Chalk Line is a device used for the creation of straight lines to aid in the construction of buildings. A length of string covered in chalk (or sometimes ink) powder is stretched across a flat surface, which is then pulled back and snapped onto the surface, leaving a straight line of the dust behind. This piece consists of three subsections entitled ``Calligraphy,'' ``Spirals,'' and ``Afterimage'' respectively.
\rightskip\leftskip
\phantom{text} \hfill (G.R.E.)
\end{center}
\endgroup

\vspace*{0.25\baselineskip}

\begingroup
\begin{center}
\leftskip3.5in
Lorem ipsum dolor sit amet, consectetur adipiscing elit. Morbi neque nibh, efficitur a eros at, elementum mollis nunc. Proin consectetur magna id purus euismod tristique. Aenean sit amet pulvinar nibh. Vivamus malesuada finibus pretium. Ut ligula mauris, commodo ac magna at, vestibulum accumsan purus. Vivamus tincidunt accumsan cursus. Ut arcu diam, vestibulum eget eros ultricies, sollicitudin facilisis tortor. Quisque massa arcu, dictum eget tristique sit amet, sagittis non nisl. Mauris varius lectus non nisl vehicula, ut feugiat ante accumsan. Nunc venenatis dui consequat erat porttitor, sit amet mollis nisl pharetra. Morbi hendrerit fringilla mauris, vitae rhoncus ante facilisis sit amet. Aenean erat felis, lobortis id dui sed, tempor finibus leo.
\rightskip\leftskip
\phantom{text} \hfill (G.R.E.)
% \phantom{text} \hfill (Juanita Pineda)
\end{center}
\endgroup

\vspace*{1.25\baselineskip}

\begin{center}
\huge PERFORMANCE NOTES
\end{center}

\begingroup
\begin{center}

\leftskip2.25in
\pmb{Pitch} : \circled{1} Accidentals apply only to the pitch which they immediately precede, but persist through ties. \circled{2} Where no trilled pitch is notated, all trills should be performed as minor seconds.
\rightskip\leftskip
\phantom{text} \hfill \phantom{()}

\vspace*{0.25\baselineskip}

\leftskip2.25in
\pmb{Grace Notes} : Grace notes should be played as fast as possible. This means that some grace figures may require a slower speed than others. All grace figures should occur before the beat, allowing the following note's attack to occur as written. As such, grace figures will occasionally shorten the duration of the note which precedes them.
\rightskip\leftskip
\phantom{text} \hfill \phantom{()}

\vspace*{0.25\baselineskip}

\leftskip2.25in
\pmb{Alternate Timbres} : Bisbigliando is notated as a trill preceded by the abbreviation \textit{Bis.} while rhythmicized timbre alterations are notated as a circled number above a note (such as \circled{1}, \circled{2}, or \circled{3}), where higher numbers refer to a greater deviation in timbre and pitch.
\rightskip\leftskip
\phantom{text} \hfill \phantom{()}

\vspace*{0.25\baselineskip}

\leftskip2.25in
\pmb{Time} : \circled{1} Accelerandi and Ritardandi are written as transitioning metronome marks. \circled{2} No pause should be taken to seperate the sections which are divided with double bars, but should be performed one directly into the next.
\rightskip\leftskip
\phantom{text} \hfill \phantom{()}

\vspace*{0.25\baselineskip}

\leftskip2.25in
\pmb{Multiphonics} : The included fingering charts are derived from Carin Levine's \textit{Die Spieltechnik der Fl\"ote}. The written pitches are desired and alternate fingerings may be employed as necessary.
\rightskip\leftskip
\phantom{text} \hfill \phantom{()}

\vspace*{0.25\baselineskip}

\leftskip2.25in
\pmb{Miscellaneous} : \circled{1} Triangle note heads indicate \textit{pizzicato}. \circled{2} Slash noteheads indicate a tone color that contains mostly air sound and some pitch while \circled{3} the hybrid notehead indicates equal parts air and pitch. \circled{4} The performer is at liberty to slur any notes with the exception of those with distinct articulations.
\rightskip\leftskip
\phantom{text} \hfill \phantom{()}
\end{center}
\endgroup

\vspace*{15\baselineskip}

\begin{center}
c. 4'30''
\end{center}

\end{document}
